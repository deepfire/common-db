% Created 2009-10-15 Thu 07:48
\documentclass[11pt]{article}
\usepackage[utf8]{inputenc}
\usepackage[T1]{fontenc}
\usepackage{graphicx}
\usepackage{longtable}
\usepackage{soul}
\usepackage{hyperref}


\title{COMMON-DB: отладчик для кристаллов семейства Мультикор.}
\author{Samium Gromoff}
\date{15 October 2009}

\begin{document}

\maketitle

\setcounter{tocdepth}{3}
\tableofcontents
\vspace*{1cm}

\begin{center}
Замечания по использованию. \\
\\
October 15, 2009, 05:51
\end{center}


\section{Замечания по использованию}
\label{sec-1}

\subsection{Параметры командной строки}
\label{sec-1.1}

\begin{itemize}
\item интерпретация параметров командной строки осуществляется примерно следующим образом:

\begin{itemize}
\item всё что имеет форму \''quoted like this\'' воспринимается как строка,
\item всё что похоже на целое число воспринимается как шестнадцатиричное,
\item иначе объект воспринимается как символ\footnote{Символ в контексте языка Common Lisp, а проще говоря, либо
\end{itemize}

\end{itemize}
символьная константа, вроде t/nil/pi (истина/ложь/3.14\ldots{}), либо
значение соответствующее названной переменной. }

\subsection{Командный интерфейс}
\label{sec-1.2}

\begin{itemize}
\item команда `scan' необходима для поиска присоединённых устройств
\item пути необходимо указывать в следующей форме:

\begin{verbatim}
       "/path/to/foo", в UNIX-системах, или:
       "d:/path/to/foo", в системах Windows.
\end{verbatim}

\item ввод чисел по основанию 16 осуществляетс с помощью следующего синтаксиса:

\begin{verbatim}
       #xdeadbeef
\end{verbatim}

\item словный доступ к памяти осуществляется следующим образом: 

\begin{verbatim}
       (setf (interface:interface-bus-word *interface* #x0) #xdeadbeef) -- запись #xdeadbeef по физическому адресу 0
       (interface:interface-bus-word *interface* #x0)  -- чтение физического адреса 0
\end{verbatim}


        в качестве замечания:

\begin{verbatim}
       (setf (interface:interface-bus-word *interface* #x0) #xdeadbeef
             (interface:interface-bus-word *interface* #x4) #xdeadbeef
             (interface:interface-bus-word *interface* #x8) #xdeadbeef)
\end{verbatim}


        .. более приятный синтаксис для группы записей.
\item регистр не имеет значения нигде\footnote{Почти. В данном контексте пользователь вряд ли столкнётся с
\end{itemize}
\href{http://www.lispworks.com/documentation/HyperSpec/Body/02_b.htm#multiple_escape}{пунктом 6 статьи 2.2 спецификации ANSI Common Lisp}. } кроме строк:

\begin{verbatim}
       "/foo/bar" != "/Foo/Bar", однако BREAK == BreaK
\end{verbatim}


\begin{itemize}
\item по умолчанию \texttt{*PRINT-BASE*}, переменная задающая основание определяющее печать
      чисел, имеет значение равное 10; установить её значение в 16 можно следующим
      образом:

\begin{verbatim}
       (setf *print-base* #x10)
\end{verbatim}

\item при возникновении исключительных ситуаций (в т.ч. ошибок) активируется отладчик CL,
      в котором также можно исполнять команды, а также активировать т.н. \emph{рестарты} --
      реакции заготовленные на случай возникновения ошибки в данном контексте. Активация
      рестарта позволяет продолжить исполнение тем или иным образом. Выбор рестартов осуществляется
      вводом соответствующего ему номера, либо сокращения его имени префиксированного `:'
\end{itemize}
\begin{verse}
\emph{0-9} - выбрать соответствующий рестарт\\
\emph{:a}  - выбрать самый верхний рестарт с именем начинающимся на A\\
\end{verse}

\begin{itemize}
\item присутствует команда \textbf{help}, однако она не очень полезна на данный момент
\item полезные команды:

\begin{description}
\item [apropos] \emph{пример: apropos `break} \\
                   вывести перечень команд с именем содержащим `break'
\item [describe] \emph{пример: describe #'hw-break} \\
                    вывести информацию о команде `hw-break', включая возможную документацию
\end{description}

\end{itemize}

\end{document}
